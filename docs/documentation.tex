\documentclass{article}

% Language setting
% Replace `english' with e.g. `spanish' to change the document language
\usepackage[english, russian]{babel}

% Set page size and margins
% Replace `letterpaper' with `a4paper' for UK/EU standard size
\usepackage[letterpaper,top=2cm,bottom=2cm,left=3cm,right=3cm,marginparwidth=1.75cm]{geometry}

% Useful packages
\usepackage{amsmath}
\usepackage{graphicx}
\usepackage{indentfirst}
\usepackage{listings}
\usepackage{float}
\usepackage[colorlinks=true, allcolors=blue]{hyperref}

\title{Документация по использованию конвертера
EBNF-to-CFG}
\author{Емельяненко Д.С.\\ Яровикова А.С.}
\date{Декабрь 2022}

\begin{document}
\maketitle

%\begin{abstract}
%Your abstract.
%\end{abstract}

\section{Введение}
Конвертер \href{https://github.com/ynastt/EBNF-to-CFG}{EBNF-to-CFG} используется для построения \href{https://ru.wikipedia.org/wiki/Контекстно-свободная_грамматика}{КС-грамматики}, эквивалентной введенной \href{https://ru.wikipedia.org/wiki/Расширенная_форма_Бэкуса_—_Наура}{РБНФ} в пользовательском синтаксисе. 

\section{Использование}

\subsection{Установка и запуск}
\begin{enumerate}
    \item Клонируем репозиторий проекта в выбранную папку
    \begin{lstlisting}
    git clone https://github.com/ynastt/EBNF-to-CFG.git
    \end{lstlisting}
    \item Переходим в папку проекта
    \begin{lstlisting}
    cd EBNF-to-CFG
    \end{lstlisting}
    \item В папке tests создаем папку для теста и создаем файлы syntax.txt (опционально), input.txt, CFGsyntax.txt (опционально)
    \begin{lstlisting}
    cd tests
    mkdir test1
    cd test1
    vim syntax.txt
    vim input.txt
    vim CFGsyntax.txt
    \end{lstlisting} 
    \item Запускаем программу main.cpp из корневой папки проекта
    \begin{lstlisting}
    g++ -Wall -o main main.cpp
    ./main
    \end{lstlisting}
    \item Результат программы будет в терминале и созданном файле output.txt
    \begin{lstlisting}
    cd EBNF-to-CFG\tests\test1
    vim output.txt
    \end{lstlisting}
\end{enumerate}

\subsection{Формат входных данных}
Программа обрабатывает входные данные следующих файлов:
\begin{itemize}
\item syntax.txt - файл с пользовательскими параметрами РБНФ грамматики;
\item input.txt - файл с пользовательской грамматикой;
\item CFGsyntax.txt - файл с пользовательскими параметрами для КС-грамматики 
\end{itemize}

\begin{figure}[H]
Входная РБНФ грамматика имеет вид:
\centering
        $$\begin{array}{l}
        [Grammar] \to [Rules] \\  
        
        [Rules] \to [Rule] \;|\; [Rule]\,[Delim]\,[Rules] \\

        [Rule] \to [Nterm]\,[Arrow]\,[RightPart]  \\

        [RightPart] \to [Operation[]Start]\,[Term]\,[ConcOrAlt]\,[Rightpart]\,          [Operation[]End]\,[ConcOrAlt]\,[Rightpart] \;|\; \\
        
        \qquad\qquad\qquad[Operation[]Start]\,[Nterm]\,[ConcOrAlt]\,[Rightpart]\,[Operation[]End]\,[ConcOrAlt]\,[Rightpart] \;|\; \\
        
        \qquad\qquad\qquad[Operation[]Start]\,[Rightpart]\,[Operation[]End]\,[ConcOrAlt]\,[Rightpart] \;|\; \\
        
        \qquad\qquad\qquad[Operation()Start]\,[Term]\,[ConcOrAlt]\,[Rightpart]\,[Operation()End]\,[ConcOrAlt]\,[Rightpart] \;|\; \\
        
        \qquad\qquad\qquad[Operation()Start]\,[Nterm]\,[ConcOrAlt]\,[Rightpart]\,[Operation()End]\,[ConcOrAlt]\,[Rightpart] \;|\; \\
        
        \qquad\qquad\qquad[Operation()Start]\,[Rightpart]\,[Operation()End]\,[ConcOrAlt]\,[Rightpart] \;|\; \\
        
        \qquad\qquad\qquad[Operation\{\}Start]\,[Term]\,[ConcOrAlt]\,[Rightpart]\,[Operation\{\}End]\,[ConcOrAlt]\,[Rightpart] \;|\; \\
        
        \qquad\qquad\qquad[Operation\{\}Start]\,[Nterm]\,[ConcOrAlt]\,[Rightpart]\,[Operation\{\}End]\,[ConcOrAlt]\,[Rightpart] \;|\; \\
        
        \qquad\qquad\qquad[Operation\{\}Start]\,[Rightpart]\,[Operation\{\}End]\,[ConcOrAlt]\,[Rightpart] \;|\; \\
        
        \qquad\qquad\qquad[Term]\,[ConcOrAlt]\,[Rightpart] \;|\; \\
        
        \qquad\qquad\qquad[Nterm]\,[ConcOrAlt]\,[Rightpart] \;|\; \\
        
        \qquad\qquad\qquad[Empty] \\
        
        [ConcOrAlt] \to [Concat] \;|\; [Alternative] \\
        
        [Term] \to [TermStart]\,[TermStr]\,[TermEnd] \\
        
        [Nterm] \to [NtermStart]\,[NtermStr]\,[NtermEnd] \\ 
        
        [Alternative] \to \;'\;|\;'\; \\

        [Concat] \to \\
        
        [NtermStr] \to [A-Z] \\
        
        [TermStr] \to [a-z] \\
        
        [Operation[]Start] \to \;'\;[\;'\; \\
        
        [Operation[]End] \to \;'\;]\;'\; \\
        
        [Operation\{\}Start] \to \;'\; \{ \;'\; \\
        
        [Operation\{\}End] \to \;'\; \} \;'\; \\ 
        
        [Operation()Start] \to \;'\;(\;'\; \\ 
        
        [Operation()End] \to  \;'\;)\;'\; \\
        \end{array}$$  
\end{figure}
Все параметры, которые встречаются в грамматике, но не имеют правил раскрытия, пользователь может задать самостоятельно в файле syntax.txt. По умолчанию программа задаёт эти параметры, опираясь на встроенный синстаксис. Ниже перечислены данные параметры. Пустые ячейки в таблице означают то, что значение параметра = пустая строка.

\begin{table}[h!]
\centering
\begin{tabular}{ |c | c | c| }
\hline 
Название параметра & Применение	& Значение по умолчанию \\ [0.5ex]
\hline
[Delim]	& разделитель правил грамматики	& '\backslash\,n' \\ [0.5ex]
\hline
[Arrow]	& разделитель левой и правой части правила	& '->' \\ [0.5ex]
\hline
[TermStart]	& символ начала терминала &  \\ [0.5ex]
\hline
[TermEnd] & символ начала терминала &  \\ [0.5ex]
\hline
[NtermStart] & символ начала нетерминала & ' \\ [0.5ex]
\hline
[NtermEnd] & символ конца нетерминала & ' \\ [0.5ex]
\hline
[Empty] & символ пустоты (обозначение для ε) & '\#' \\ [0.5ex]
\hline
\end{tabular}
\caption{Параметры пользовательской РБНФ грамматики.}
\end{table}

\newline
\textbf{Пользовательская РБНФ грамматика имеет следующие ограничения (при несоблюдении программа аварийно завершится):}

\begin{itemize}
\item Пользователь должен вводить грамматику в input.txt, учитывая значения по умолчанию для тех параметров, значения которых он оставил пустыми в файле syntax.txt.

(Например, в РБНФ грамматике по умолчанию в качестве обрамляющих символов для нетерминалов используются одинарные кавычки 'S', если пользователь не задаст эти параметры или их оставит пустым (NtermStart= и NtermEnd=), то программа подставит значения по умолчанию, т.е.
NtermStart= ' и NtermEnd= '. Если пользователь далее введет грамматику без учета значений параметров по умолчанию, то программа выдаст ошибку и завершится.)
\item Пробел запрещен для обозначения каких-либо параметров пользовательской грамматики.
\item Начальным считается нетерминал, правило переписывания которого введено первым в input.txt.
\item Параметры в syntax.txt (СПОЙЛЕР: в CFGsyntax.txt аналогично) необходимо вводить СТРОГО в соответствии с синтаксисом: 

НАЗВАНИЕ\_ПАРАМЕТРА= ЗНАЧЕНИЕ  

(т.е. название параметра без квадратных скобок, между параметром и знаком равенства нет пробела)
\end{itemize}

\subsection{Формат выходных данных}
В результате выполнения программы в директории tests/test<номер теста> будет создан файл output.txt с построенной КС-грамматикой. Также результат программы будет выведен в терминале.

\begin{figure}[H]
КС-грамматика на выходе имеет следующий вид:
\centering
        $$\begin{array}{l}
        [Grammar] \to [Rules] \\  
        
        [Rules] \to [Rule] \;|\; [Rule]\,[Delim]\,[Rules] \\

        [Rule] \to [Nterm]\,[Arrow]\,[RightPart]  \\ 
        
        [RightPart] \to [RightPart1] \;|\; [Empty]\\ 

        [RightPart1] \to [RightPart2]\,[Newoperation] \\

        [Newoperation] \to [Alternative]\,[RightPart1] \;|\; \epsilon \\
        
        [RightPart2] \to [Term] \;|\; [Nterm] \;|\; [Nterm]\,[Concat]\,[RightPart2] \;|\; [Term]\,[Concat]\,[RightPart2] \\
        
        [Term] \to [TermStart]\,[TermStr]\,[TermEnd] \\
        
        [Nterm] \to [NtermStart]\,[NtermStr]\,[NtermEnd] \\ 
        
        [NtermStr] \to [A-Z] \\
        
        [TermStr] \to [a-z] \\
        \end{array}$$  
\end{figure}
Все параметры, которые встречаются в грамматике, но не имеют правил раскрытия, пользователь может задать самостоятельно в файле CFGsyntax.txt. По умолчанию программа задаёт эти параметры, опираясь на встроенный синстаксис. Ниже перечислены данные параметры. Пустые ячейки в таблице означают то, что значение параметра = пустая строка.

\begin{table}[h!]
\centering
\begin{tabular}{ |c | c | c| }
\hline 
Название параметра & Применение	& Значение по умолчанию \\ [0.5ex]
\hline
[Delim]	& разделитель правил грамматики	& '\backslash\,n' \\ [0.5ex]
\hline
[Arrow]	& разделитель левой и правой части правила	& '->' \\ [0.5ex]
\hline
[Empty] & символ пустоты (обозначение для ε) &  \\ [0.5ex]
\hline
[TermStart]	& символ начала терминала &  \\ [0.5ex]
\hline
[TermEnd] & символ начала терминала &  \\ [0.5ex]
\hline
[NtermStart] & символ начала нетерминала & ' \\ [0.5ex]
\hline
[NtermEnd] & символ конца нетерминала & ' \\ [0.5ex]
\hline
[Concat] & символ конкатенации &  \\ [0.5ex]
\hline
[Alternative] & символ для альтернативы & '|' \\ [0.5ex]
\hline
\end{tabular}
\caption{Параметры КС-грамматики.}
\end{table}

\subsection{Пример использования}
\textbf{syntax.txt}
\begin{lstlisting}   
    Delim= 
    Arrow= :=
    NtermStart= SS
    NtermEnd= SS
\end{lstlisting}

\textbf{input.txt}
\begin{lstlisting}   
    SSKSS := [a {SSSSS}] | b | # 
    SSSSS := c | #
\end{lstlisting}

\textbf{CFGsyntax.txt}
\begin{lstlisting}   
    Arrow= =>
    Empty= @
    Alternative= \\
\end{lstlisting}

\textbf{output.txt (и в терминале)}
\begin{lstlisting}   
    'K'=>'A0'
    'A0'=>'A2'\\'A1'
    'A1'=>b\\@
    'A2'=>a'A3'\\@
    'A3'=>'S''A3'\\@
    'S'=>'A4'
    'A4'=>c\\@
\end{lstlisting}

\end{document}